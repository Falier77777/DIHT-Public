\documentclass[11pt; a4paper]{report}
\usepackage{graphicx} %for pic
\usepackage[utf8x]{inputenc}
\usepackage[english, russian]{babel} % for rus

%\usepackage{amsbsy}

\usepackage{fancyhdr}  % for beauty hat, floor and sides style
\pagestyle{fancy}
\fancyheadoffset{1.5 cm}  % increase Headrule length
\renewcommand{\headrulewidth}{0.25mm}
%\renewcommand{\sectionmark}[1]{\markboth{#1}{}}
%\lhead{\thepage}

%%% Страница
\usepackage{geometry} % Простой способ задавать поля
\geometry{top=50mm}
\geometry{bottom=45mm}
\geometry{left=45mm}
\geometry{right=45mm}
%
\usepackage{xcolor}  % for href and href colour
\usepackage{hyperref}

\usepackage{amsmath} 
\usepackage{ mathrsfs } %for beauty math alphabet

%%%%%% theorems
\usepackage{amsthm}  % for theoremstyle

\theoremstyle{plain} % Это стиль по умолчанию, его можно не переопределять.
\newtheorem{theorem}{Theorem}
\newtheorem{lemma}{Lemma}[section]
\newtheorem{proposition}[theorem]{Утверждение}
\newtheorem{sug}{Предположение}[section]

\theoremstyle{defenition} 
\newtheorem{glob_def}{Defenition}

\theoremstyle{remark}
\newtheorem*{small_def}{def}
\newtheorem*{exmpl}{Example}
%%%%%%%%

\usepackage[normalem]{ulem}  % to make text crossed out with \sout{} command

\begin{document} 

\let\ms\mathscr  % let use \ms instead of \mathscr 
\let\ol\overline  % (one more \overline{\mathscr{L}} and I maybe kill myself)
\let\bs\boldsymbol
\let\ra\rightarrow
\let\b\begin
\let\e\end
\let\it\item
\let\subs\subsection
%%%%%%%%

\large


\title{Линейная Алгебра\\ 2 семестр}
\author{Фреик Александр Андреевич} 
\date{Весна 2020}
\maketitle
Это конспект учебника по аналитеческой геометрии от Беклимешева Д. В. для подготовки к весеннему экзамену 1 курса по линейной алгебре.\\ 
P.S.  Он не пустой, промотайте вниз, пожалйста =)
\tableofcontents

\newpage
\chapter{Матрицы и системы линейных уравнений}
\section{\Large 1 Пункт}
\subsection{Ранг матрицы}
\subsection{Теоремы о базисном миноре}
\subsection{Теорема о ранге матрицы}

\newpage
\section{\Large 2 Пункт}
\subsection{Системы линейных уравнений}
\subsection{Метод Гаусса}
\subsection{Теорема Кронекера-Капелли}
\subsection{Фундаментальная система решений и общее решение однородной системы линейных уравнений}
\subsection{Общее решение неоднородной системы}
\subsection{Теорема Фредгольма}


\chapter{Линейные пространства}
\section{\Large 3 Пункт}
\subsection{Аксиоматика линейного пространства}
Определение. Множество $\mathscr{L}$ мы назовем линейным пространством, а его элементы — векторами, еcли:
\b{enumerate}
\item Задан закон (операция сложения) по которому любым двум элементам х и у из  $\mathscr{L}$ сопоставляется элемент, называемый их суммой и обозначаемый 
\item Задан закон (операция умножения на число), по которому элементу х из $\mathscr{L}$ и числу $\alpha$ сопоставляется элемент из $\mathscr{L}$, называемый произведением х на а и обозна­чаемый а х .
\item Для любых элементов х, у и г из для любых чисел $\alpha$ и $\beta$ выполнены следующие требования (или аксиомы):
\begin{enumerate}
\item $x+y = y+x$
\item $(x + y) + z = x + (y + z)$
\item Существует элемент 0 такой что для каждого х из $\mathscr{L}$ выполнено $x + 0=x$
\item Для каждого х существует элемент -x такой, что $x + (-x) = 0$
\item $\alpha(x+y) = \alpha x + \alpha y$
\item $ (\alpha + \beta)x = \alpha x+\beta y$
\item $ \alpha( \beta x) = (\alpha \beta) x$
\item $1x = x$
\end{enumerate}
\e{enumerate}

Если в п. 2)мы ограничиваемся вещественными чис­лами, то $\mathscr{L}$ называется вещественным линейным прост­ранством, если же определено умножение на любое комплексное число, то линейное пространство называется комплексным.
Вектор -х называется противоположным вектору х.
Вектор 0 называется нулевым вектором или нулем.

\subsection{\Large Линейная зависимость и линейная независимость систем элементов в линейном пространстве.}
По аналогии с соответствую­щими определениями для векторов и для столбцов, веденными в гл. I и V, мы можем определить линейно зависимую и линейно независимую систему векторов в линейном пространстве. Напомним, что линейная комбина­ция называется тривиальной, если все ее коэффициенты равны нулю.\\

\begin{glob_def}
Система векторов называется линейңо зависимой, если существует равная нулю нетривиальная линейная комбинация этих векторов. В против­ном случае, т. е, когда только тривиальная линейная ком­бинация векторов равна нулю,  система векторов назы­вается линейно независимой. \\
\end{glob_def}

О линейно зависимых и линейно независимых систе­мах векторов справедливы те же предложения, что и о таких же системах столбцов.. Мы приведем здесь только формулировки, так как доказательства не отличаются от доказательств предложений о столбцах (см. предложения 2—5 § 1 гл. V). \\
\begin{sug}\label{6.1}
Система из k > 1 векторов линейно зависима тогда и только тогда, когда хотя бы один из векторов есть линейная комбинация остальных.\\
\end{sug}

\begin{sug}\label{sug6.2}
Если в систему входит нулевой вектор, то она линейно зависима.\\
\end{sug}

\begin{sug}\label{sug6.3}
Если в~систему входит нулевой вектор, то она линейно зависима.\\
\end{sug}

\begin{sug}\label{sug6.4}
Каждая подсистема линейно независимой системы векторов сама линейно независима.\\
\end{sug}

\begin{sug}\label{sug6.5}
Если вектор раскладывается по линейно неза­висимой системе векторов, то коэффициенты разложения определены однозначно.\\
\end{sug}

\subsection{Базис и размерность.}
\begin{glob_def}
Базисом в линейном пространстве $\mathscr{L}$  мы назо­вем упорядоченную конечную систему векторов, если:
\begin{itemize}
\item она линейно независима
\item каждый вектор из $\mathscr{L}$   раскладывается в линейную комбинацию векторов этой системы.
\end{itemize}
\end{glob_def}

В определении сказано, что базис \textemdash\  упорядоченная система век­ торов. Это означает,
 что из одного и того же множества векторов можно составить разные базисы, по-разному нумеруя векторы.
Коэффициенты линейной комбинации, о которой идет речь
в определении базиса, называются компонентами или координатами вектора в данном базисе.\\
Векторы базиса $e_1, \ldots, e_n$ мы будем записывать в виде строки, 
а компоненты $\xi_1, \ldots, \xi_n$ вектора в базисе $\boldsymbol{e}$ -- в столбец $\begin{bmatrix} \xi_1 \\  \vdots \\ \xi_n \end{bmatrix}$ который назовем координатным столбцом вектора.\\
Разложение вектора по  базису:
 \begin{equation*}
 x = \sum \xi^ie_i =  \begin{bmatrix} p_1 &  e_2 & p_3 \end{bmatrix}  
 \begin{bmatrix} \xi_1 \\  \xi_2 \\ \xi_3 \end{bmatrix} = \boldsymbol{ e\xi}
 \end{equation*}
Из предложения \ref{sug6.5} непосредственно следует, что компоненты век­ тора в данном базисе определены однозначно.

\begin{sug}\label{sug6.6}
Координатный столбец суммы векторов ра­ вен сумме их координатных столбцов. Координатный столбец произ­ ведения вектора на число равен произведению координатного столбца данного вектора на это число.
\end{sug}
\begin{proof}
Для доказательства просто перемножим строку базиса с координатными столбцами и воспользуемся свойствами матриц.
\end{proof}

\begin{sug}\label{sug6.7}
Векторы линейно зависимы тогда и только тогда, когда линейно зависимы их координатные столбцы.
\end{sug}

\begin{sug}\label{sug6.8}
Если в линейном пространстве существует базис из n векторов, то любая система из m > n векторов линейно
зависима.
\end{sug}
\begin{proof}
Предположим, что в пространстве сущест­вует базис $e_1, \ldots, e_n$ , и рассмотрим систему векторов 
$ f_i , \ldots, f_m$, при­ чем m > n. Каждый из векторов $f_i , \ldots, f_m$ мы разложим по базису и составим матрицу из их координатных столбцов. Это матрица раз­меров m х n, и ранг ее не превосходит n. Поэтому столбцы матрицы линейно зависимы, а значит, линейно зависимы и векторы $ f_i , \ldots, f_m$
\end{proof}
\begin{theorem}\label{t6.1}
Если в линейном пространстве есть базис из n век­ торов, то и любой другой базис состоит из n векторов.
\end{theorem}

\begin{glob_def}
Линейное пространство, в котором существует базис из n векторов, называется n-мерным, а число n - размерностью пространства. Размерность пространства $\mathscr{L}$ обозначается $dim\mathscr{L}$
\end{glob_def}

В нулевом пространстве нет базиса, так как система из одно­ го нулевого вектора линейно зависима. 
Размерность нулевого прост­ранства по определению считаем равной нулю.
Может случиться, что каково бы ни было натуральное число m, в пространстве найдется m линейно
 независимых векторов. Такое пространство называется бесконечномерным. Базиса в нем не сущест­вует: 
 если бы был базис из n векторов, то любая система из n + 1 векторов была бы линейно зависимой по 
 предложению \ref{sug6.8}.
 \begin{exmpl}
 Линейное пространство функций от одной перемен­ной t, определенных и непрерывных на отрезке [0, 1] 
 является бес­конечномерным. Чтобы это проверить, достаточно доказать, что при любом m в нем существует 
 линейно независимая система из m век­торов. Зададимся произвольным числом m. Векторы нашего 
 прост­ранства - функции $t_0 = 1, t, t^2 ,\ldots, t^{m-1}$ -- линейно независимы. Действительно, 
 равенство нулю линейной комбинации этих векторов означает, что многочлен
$\alpha_0+ \alpha_1t+ \alpha_2t^2 +\ldots+\alpha_{m-1}t^{m-1}$
В n-мерном пространстве каждая упорядо­ченная линейно независимая система из n векторов есть базис.
\end{exmpl}

\begin{sug}\label{sug6.9}
Если в линейном пространстве существует базис из n векторов, то любая система из m > n векторов линейно
зависима.
\end{sug}
\begin{proof}
Очевидно.
\end{proof}

\begin{sug}\label{sug6.10}
В n-мерном пространстве каждую упорядоченную линейно независимую систему из k < n векторов можно 
до­полнить до базиса.
\end{sug}
\begin{proof}
Очевидно.
\end{proof}


\newpage
\section{4 Пункт}

\subsection{Матрица перехода от одного базиса к другому}

\textbf{Замена базиса.} Если в п-мерном пространстве даны два ба­ зиса  $e_1, \ldots, e_n$ и $e_1', \ldots, e_n'$, то мы можем разложить каждый вектор вто­рого базиса по первому базису:
\begin{equation}\label{eq1}
e'_i = \sum \sigma_i^je_j \qquad (i = 1, \ldots, n)
\end{equation}
Компоненты $\sigma_i^j$; можно записать в виде квадратной матрицы 
\begin{equation*}
S =  \begin{bmatrix}
		\sigma_{11} & \sigma_{12} &  \ldots & \sigma_{1n} \\
		\sigma_{21} & \sigma_{22} &  \ldots & \sigma_{2n} \\
		\vdots & \vdots & \ddots & \vdots \\
		\sigma_{n1} & \sigma_{n2} & \ldots & \sigma_{nn} 
	\end{bmatrix} 
\end{equation*}
Столбцы этой матрицы -- координатные столбцы векторов $e_1', \ldots, e_n'$ в базисе $\boldsymbol{e}$. Поэтому столбцы линейно независимы, и $det S \not= 0$.

\begin{glob_def}
Матрицу, j-й столбец которой есть координат­ный столбец вектора $e_j$ в базисе е, мы назовем матрицей перехода от базиса е к базису $е'$.
\end{glob_def}
Равенство \ref{eq1} можно переписать в матричных обозначениях: 
\begin{equation*}
\begin{bmatrix} e'_1 &  \ldots & e'_n \end{bmatrix} = \begin{bmatrix} e_1 &  \ldots & e_n \end{bmatrix} S
\end{equation*}
или 
\begin{equation}\label{eq6.1.2}
\boldsymbol{e'} = \boldsymbol{e} S
 \end{equation}
Пусть в линейном пространстве даны три базиса и s and T матрицы перехода от 1 ко 2му и 2го к 3му
\begin{equation}\label{eq6.3}
\boldsymbol{e''} = \boldsymbol{e} S T
\end{equation}
 
\begin{sug}\label{sug6.11}
Пусть задан базисе. Каждая матрица S с $det S \not=0$ есть матрица перехода от е к некоторому базису е'.\end{sug}
\begin{proof}
Очевидно.
\end{proof}

\subsection{Изменение координат при изменении базиса в линейном пространстве.}
Cвязь компонент одного и того же векторах в двух разных базисах. Пусть $x = \boldsymbol{e\xi} = 
 \boldsymbol{e'\xi'} = \boldsymbol{eS\xi'}$. Итак, мы имеем разложение векторах по базису е в двух видах, и в силу единственности координатного столбца получаем: 
 \begin{equation}\label{eq6.4}
  \boldsymbol{\xi} = S\boldsymbol{\xi'}
 \end{equation}
 
 
\subsection{Координатное представление векторов линейного пространства и операций с ними.}
Нужно сказать про координатные столбцы и операции над ними будут как обычные операции над матрицами.
(В "обычном" векторном пространстве)

\subsection{Ориентация пространства}
Понятие ориентации прямой, плоскости и пространства в §4 гл. 1 основывалось на разделении всех базисов на два класса. Произведем это разделение для вещественных линейных пространств любой размерности.

Фиксируем некоторый базис е0 и обозначим через $\mathscr{E}_+(e_0)$ мно­жество всех таких базисов е, 
что $\boldsymbol{e} = \boldsymbol{e}_0 S, detS>0$. Остальные базисы отнесем к классу $\mathscr{E}_-(e_0)$. 
Ясно, что для $\boldsymbol{e'}  \in \mathscr{E}_+(e_0)$ выполне­но $\boldsymbol{e'} = \boldsymbol{e}_0 T, detT>0$.

\begin{sug}\label{sug6.12}
Классы базисов $\mathscr{E}_+(e_0)$ и  $\mathscr{E}_-(e_0)$ не зависят от выбора исходного базиса е0.
\end{sug}
\begin{proof}
Очевидно. т.к. $detST = detS *detT$
\end{proof}

\begin{glob_def}
Вещественное линейное пространство называет­ ся ориентированным, если из двух классов базисов 
$\mathscr{E}_+(e_0)$ и  $\mathscr{E}_-(e_0)$ указан один. Базисы выбранного класса называются положительно
ориенти­рованными.
\end{glob_def}


\subsection{Теорема об изоморфизме.}
Пока пропускаю




\newpage
\section{\Large 5 Пункт}
\subsection{Подпространства и способы их задания в линейном пространстве. }
\begin{glob_def}
Непустое подмножество $\mathscr{L'}$ векторов линейного пространства $\mathscr{L}$ называется линейным подпространством, если:
    \begin{itemize}
    \item сумма любых векторов из $\mathscr{L'}$ принадлежит $\mathscr{L'}$
    \item произведение каждого вектора из  $\mathscr{L'}$ на любое число также принадлежит  $\mathscr{L'}$
    \end{itemize}
\end{glob_def}
Подпространство является линейным про­странством.

\begin{glob_def}
Пусть дано некоторое множество $\mathscr{P}$ векторов в линей­ ном пространстве $\mathscr{L}$. Линейной оболочкой $\mathscr{L}$ множества $\mathscr{P}$ называется множество $\mathscr{L'}$ всех линейных комбинаций векторов множества $\mathscr{P}$. 
\end{glob_def}

В частности, если $\mathscr{P}$ - конечное множество векторов, мы имеем
\begin{sug}\label{sug6.2.1}
Размерность линейной оболочки множества из m векторов не превосходит m.
\end{sug}
\begin{proof}
Очевидно. 
\end{proof}

\begin{sug}\label{sug6.2.2}
Пусть $\mathscr{L'}$ -- подпространство n-мерного пространства $\mathscr{L}$. Тогда $dim\mathscr{L'} \leq n$. Если $dim\mathscr{L'}=n$, то $\mathscr{L}$ сов­падает с $\mathscr{L'}$.
\end{sug}
\begin{proof} Очевидно. \end{proof}

\begin{sug}\label{sug6.2.3}
Пусть $\mathscr{L'}$ -- подпространство n-мерного пространства $\mathscr{L}$.
 Если базис $e_1, \ldots , e_k$ в $\mathscr{L'}$ дополнить до ба­зиса $e_1, \ldots , e_k, \ldots , e_n$ в $\mathscr{L}$, то в таком базисе все векторы из  $\mathscr{L}$ и только они будут иметь компоненты $\xi^{k+1} = \ldots = \xi^n = 0$.
\end{sug}
\begin{proof} Очевидно. \end{proof}

\begin{sug}\label{sug6.2.4}
Пусть в n -- мерном пространстве $\mathscr{L}$ выбран базис. Тогда координатные 
столбцы векторов, принадлежащих k-мер­ному подпространству $\mathscr{L'}$ (k < n),
удовлетворяют однородной систе­ ме линейных уравнений ранга n - k.
    \begin{equation*}
    \xi^{j} = \sum_{i=1}^n \sigma_i^{j}\xi'^i = 0 \qquad (j = k+1, \ldots, n)
    \end{equation*}
\end{sug}
\begin{proof} Действительно, при замене базиса старые компоненты выражаются
через новые по формулам \ref{eq6.1.2}, и в новом базисе система уравнений примет такой вид.
\end{proof}





\subsection{Сумма и пересечение подпространств}
Рассмотрим два подпространства $\mathscr{L'}$ и $\mathscr{L''}$ линейного пространства $\mathscr{L}$
\begin{glob_def}
Будем называть суммой подпространств $\mathscr{L'}$ и $\mathscr{L''}$ и обозначать
 $\mathscr{L'}+\mathscr{L''}$ линейную оболочку их объединения $\mathscr{L'} \cup \mathscr{L''}$.
\end{glob_def}
Пусть размерности подпространств $\mathscr{L'}$ и $\mathscr{L''}$ равны k и l.
 Выберем в этих подпространствах базисы $e_1, \ldots, e_k$ и $f_1, \ldots, f_l$. 
 Каждый вектор из $\mathscr{L'}$ и $\mathscr{L''}$ раскладывается по векторам 
 $e_1, \ldots, e_k, f_1, \ldots, f_l$, и мы по­ лучим базис в $\mathscr{L'}+\mathscr{L''}$, 
 если удалим из этой системы все векторы, которые линейно выражаются через остальные.
 Сделать это можно, например, так:\\
 Выберем какой-либо базис в $\mathscr{L}$ и составим матрицу из координат­ных столбцов всех векторов 
$e_1, \ldots, e_k, f_1, \ldots, f_l$. Те векторы, координат­ые столбцы которых - базисные столбцы этой матрицы, 
составляют базис в  $\mathscr{L'}+\mathscr{L''}$.

\begin{glob_def}
Назовем пересечением подпространств $\mathscr{L'}$ и $\mathscr{L''}$ и обозначим
 $\mathscr{L'} \cap \mathscr{L''}$ множество векторов, которые принадлежат обо­им подпространствам.\end{glob_def}
Пересечение $\mathscr{L'}$ и $\mathscr{L''}$ есть подпространство. \\
Для суммы s > 2 пространств
\begin{equation*}
dim2( \mathscr{L^1}+\ldots+\mathscr{L^s}) \leq dim\mathscr{L^1}+\ldots+dim\mathscr{L^s}
\end{equation*}



\subsection{Прямая сумма}
\begin{glob_def}
Сумма подпространств $\mathscr{L^1}+\ldots+\mathscr{L^s}$ называется прямой суммой,
 если ее размерность равна сумме размерностей этих подпространств, 
 т. е. имеет максимальное из возможных значений.
\end{glob_def}
Если надо подчеркнуть в обозначении, что сумма прямая, то ис­пользуют знак $\oplus$.\\

Прибавление нулевого подпространства не меняет ни размерность суммы, ни сумму размерностей.
 Но ниже мы будем считать подпро­ странства ненулевыми, чтобы избежать оговорок, вызванных несу­ ществованием базиса в нулевом подпространстве.

\begin{sug}\label{sug6.2.5}
Для того чтобы сумма  $\mathscr{L'}$ подпространств $\mathscr{L^1}+\ldots+\mathscr{L^s}$ 
была прямой суммой, необходимо и достаточно выполнение любого из следующих четырех свойств: 
   \begin{enumerate}
    \item любая система из $m\leq s$ ненулевых векторов, принадлежащих различным подпространствам 
     $\mathscr{Li}^ i( = 1, ..., s)$, линейно независима
    \item каждый вектор $x  \in \mathscr{L'}$ однозначно раскладывается в сумму $x_1+...+x_8$, 
    где $x_i \in  \mathscr{L}^i\qquad i( =1,...,s)$
    \item пересечение каждого из подпространств $\mathscr{L}^i$ с суммой остальных есть 
    нулевое подпространство
    \item объдинение базисов подпространств $\mathscr{L}^i$ (i = 1, ..., s) есть базис в $\mathscr{L'}$
    \end{enumerate}
\end{sug}

\begin{proof} 
стр 169
Опр -> 1 от противного:
Допустим, что нашлась линейно зависимая система ненулевых век­торов таких, что никакие два из них не лежат в одном и том же подпространстве. Дополним каждый из этих векторов до базиса в его подпространстве, а в тех подпространствах, из которых в системе векторов нет, выберем базис произвольно. Объединение этих базисов- л.з. система.

 1 -> 2  от противного. 
 Допустим, что б) не выполнено и некоторый вектор х представлен как сумма xi и yi. 
 Тогда (х1 - y1) + ... + (xs - ys) = 0, и каждая разность так же явдяется вектором 
 соответствуюзего пространства => Если хоть одна из разностей от­лична от нуля, 
 мы получаем противоречие со свойством a).
 2->3 так же от противного
 3->4 так же от противного
\end{proof}

Легко видеть, что при сложении подпространств можно произ­вольно расставлять и убирать скобки.

\begin{sug}\label{sug6.2.6}
Для любого подпространства $\mathscr{L}'$ пространст­ва $\mathscr{L}$ найдется 
такое подпространство $\mathscr{L}''$, что $\mathscr{L}=\mathscr{L}' \oplus \mathscr{L}''$ 
\end{sug}
\begin{proof} 
 Выберем базис е1 , ..., ek подпространства' и дополним его до базиса пространства векторами ek+1, ..., еn,
  Ли­нейную оболочку ek+l, ..., еn обозначим через ". Из предложения 5 видно, что
  $\mathscr{L}=\mathscr{L}' \oplus \mathscr{L}''$ 
\end{proof}



\subsection{Формула размерности суммы подпространств}
\begin{theorem}\label{t2}
    Размерность суммы двух подпространств равна сум­ ме их размерностей минус размерность их пересечения.
\end{theorem}
\begin{proof} 
    Если сумма прямая, утверждение справедливо: размерность равна сумме размерностей, 
    а пересечение нулевое.\\
    Иначе $\mathscr{L}^2=\mathscr{M} \oplus (\mathscr{L}^1 \cap \mathscr{L}^2) => 
    \mathscr{L}^1+\mathscr{L}^2=\mathscr{M} + \mathscr{L}^1 = \mathscr{M} \oplus\mathscr{L}^1$
    because $z \in (\mathscr{M} \cap \mathscr{L}^1)=> since \mathscr{M} \subset \mathscr{L}^2\\
    z \in (\mathscr{L}^2 \cap \mathscr{L}^1) \cap \mathscr{M} = 0$
\end{proof}


\newpage
\section{\Large 6 Пункт}
\subsection{Линейные отображения линейных пространств и линейные преобразования линейного пространства.}
Пусть  $\mathscr{L}$ и $\overline{\mathscr{L}}$- два линейных пространства, оба вещественные
или оба комплексные. Под отображением А про­странства $\mathscr{L}$ в пространство
 $\overline{\mathscr{L}}$ понимается закон, по которому каж­дому вектору из $\mathscr{L}$ 
 сопоставлен единственный вектор из $\overline{\mathscr{L}}$
\begin{glob_def}
    Отображение $A: \mathscr{L} \rightarrow \overline{\mathscr{L}}$ называется линейным, если
     для любых векторов x и у из и $\mathscr{L}$ любого числа $\alpha$ выполнены равенства
    \begin{equation}
        A(x + y) = A(x) + A(y),\qquad A(\alpha x) = \alpha A(x)
    \end{equation}
\end{glob_def}
Линейное отображение мы будем называть линейным преобразова­нием, если пространства
 $\mathscr{L}$ и $\ol{\ms{L}}$ совпадают.
 
\begin{sug}\label{sug}
При линейном отображении $A: L \rightarrow \overline{L} $ и­  линейное подпространство переходит в линейное подпрост­ранство причем
\end{sug}
\begin{proof} 

\end{proof}


\subsection{Ядро и образ линейного отображения.}
\subsection{Операции над линейными преобразованиями}
\subsection{Ядро и образ линейного отображения.}
\subsection{Обратное преобразование}
\subsection{Линейное пространство линейных отображений (преобразований)}


\newpage
\section{Матрицы линейного отображения и линейного преобразования для конечномерных пространств. Операции над линейными преобразованиями в матричной форме. Изменение матрицы линейного отображения (пре- образования) при замене базисов. Изоморфизм пространства линейных отображений и пространства матриц.}


\newpage
\section{Инвариантные подпространства линейных преобразований. Собственные векторы и собственные значения. Собственные подпространства. Линейная независимость собственных векторов, принадлежащих различным собственным значениям.}

\end{document}

\begin{glob_def}

\end{glob_def}

\begin{sug}\label{sug}

\end{sug}
\begin{proof} 

\end{proof}


%% 	matrix example
\begin{equation*}
  \begin{matrix}
	1 & 2 \\
	 3&4 
   \end{matrix} \qquad
	 \begin{bmatrix}
		p_{11} & p_{12} &  \ldots & p_{1n} \\
		p_{21} & p_{22} &  \ldots & p_{2n} \\
		\vdots & \vdots & \ddots & \vdots \\
		p_{m1} & p_{m2} & \ldots & p_{mn} 
	\end{bmatrix} 
\end{equation*}
%%