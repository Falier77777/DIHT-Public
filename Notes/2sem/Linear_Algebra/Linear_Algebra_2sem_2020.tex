\documentclass[11pt; a4paper]{report}
\usepackage{graphicx} %for pic
\usepackage[utf8x]{inputenc}
\usepackage[english, russian]{babel} % for rus

%\usepackage{amsbsy}

\usepackage{fancyhdr}  % for beauty hat, floor and sides style
\pagestyle{fancy}
\fancyheadoffset{1.5 cm}  % increase Headrule length
\renewcommand{\headrulewidth}{0.25mm}
%\renewcommand{\sectionmark}[1]{\markboth{#1}{}}
%\lhead{\thepage}

%%% Страница
\usepackage{geometry} % Простой способ задавать поля
\geometry{top=50mm}
\geometry{bottom=45mm}
\geometry{left=45mm}
\geometry{right=45mm}
%
\usepackage{xcolor}  % for href and href colour
\usepackage{hyperref}

\usepackage{amsmath} 
\usepackage{ mathrsfs } %for beauty math alphabet

%%%%%% theorems
\usepackage{amsthm}  % for theoremstyle

\theoremstyle{plain} % Это стиль по умолчанию, его можно не переопределять.
\newtheorem{theorem}{Theorem}
\newtheorem{lemma}{Lemma}[section]
\newtheorem{proposition}[theorem]{Утверждение}
\newtheorem{sug}{Предположение}[section]

\theoremstyle{defenition} 
\newtheorem{glob_def}{Defenition}

\theoremstyle{remark}
\newtheorem*{small_def}{def}
\newtheorem*{exmpl}{Example}
%%%%%%%%


\begin{document} 
\large
\title{Линейная Алгебра\\ 2 семестр}
\author{Фреик Александр Андреевичем} 
\date{Весна 2020}
\maketitle
Это конспект учебника по аналитеческой геометрии от Беклимешева Д. В. для подготовки к весеннему экзамену 1 курса по линейной алгебре. 
\tableofcontents

\newpage
\chapter{Матрицы и системы линейных уравнений}

\section{Ранг матрицы. Теоремы о базисном миноре. Теорема о ранге матрицы.}


\section{Системы линейных уравнений. Метод Гаусса. Теорема Кронекера- Капелли. Фундаментальная система решений и общее решение однород- ной системы линейных уравнений. Общее решение неоднородной систе- мы. Теорема Фредгольма.}


\chapter{Линейные пространства}
\section{\Large 3 Пункт}
\subsection{\Large Аксиоматика линейного пространства}
Определение. Множество $\mathscr{L}$ мы назовем линейным пространством, а его элементы — векторами, еcли:
\begin{enumerate}
\item Задан закон (операция сложения) по которому любым двум элементам х и у из  $\mathscr{L}$ сопоставляется элемент, называемый их суммой и обозначаемый 
\item Задан закон (операция умножения на число), по которому элементу х из $\mathscr{L}$ и числу $\alpha$ сопоставляется элемент из $\mathscr{L}$, называемый произведением х на а и обозна­чаемый а х .
\item Для любых элементов х, у и г из для любых чисел $\alpha$ и $\beta$ выполнены следующие требования (или аксиомы):
\begin{enumerate}
\item $x+y = y+x$
\item $(x + y) + z = x + (y + z)$
\item Существует элемент 0 такой что для каждого х из $\mathscr{L}$ выполнено $x + 0=x$
\item Для каждого х существует элемент -x такой, что $x + (-x) = 0$
\item $\alpha(x+y) = \alpha x + \alpha y$
\item $ (\alpha + \beta)x = \alpha x+\beta y$
\item $ \alpha( \beta x) = (\alpha \beta) x$
\item $1x = x$
\end{enumerate}
\end{enumerate}

Если в п. 2)мы ограничиваемся вещественными чис­лами, то $\mathscr{L}$ называется вещественным линейным прост­ранством, если же определено умножение на любое комплексное число, то линейное пространство называется комплексным.
Вектор -х называется противоположным вектору х.
Вектор 0 называется нулевым вектором или нулем.

\subsection{\Large Линейная зависимость и линейная независимость систем элементов в линейном пространстве.}
По аналогии с соответствую­щими определениями для векторов и для столбцов, веденными в гл. I и V, мы можем определить линейно зависимую и линейно независимую систему векторов в линейном пространстве. Напомним, что линейная комбина­ция называется тривиальной, если все ее коэффициенты равны нулю.\\

\begin{glob_def}
Система векторов называется линейңо зависимой, если существует равная нулю нетривиальная линейная комбинация этих векторов. В против­ном случае, т. е, когда только тривиальная линейная ком­бинация векторов равна нулю,  система векторов назы­вается линейно независимой. \\
\end{glob_def}

О линейно зависимых и линейно независимых систе­мах векторов справедливы те же предложения, что и о таких же системах столбцов.. Мы приведем здесь только формулировки, так как доказательства не отличаются от доказательств предложений о столбцах (см. предложения 2—5 § 1 гл. V). \\
\begin{sug}\label{6.1}
Система из k > 1 векторов линейно зависима тогда и только тогда, когда хотя бы один из векторов есть линейная комбинация остальных.\\
\end{sug}

\begin{sug}\label{sug6.2}
Если в систему входит нулевой вектор, то она линейно зависима.\\
\end{sug}

\begin{sug}\label{sug6.3}
Если в~систему входит нулевой вектор, то она линейно зависима.\\
\end{sug}

\begin{sug}\label{sug6.4}
Каждая подсистема линейно независимой системы векторов сама линейно независима.\\
\end{sug}

\begin{sug}\label{sug6.5}
Если вектор раскладывается по линейно неза­висимой системе векторов, то коэффициенты разложения определены однозначно.\\
\end{sug}

\subsection{Базис и размерность.}
\begin{glob_def}
Базисом в линейном пространстве $\mathscr{L}$  мы назо­вем упорядоченную конечную систему векторов, если:
\begin{itemize}
\item она линейно независима
\item каждый вектор из $\mathscr{L}$   раскладывается в линейную комбинацию векторов этой системы.
\end{itemize}
\end{glob_def}

В определении сказано, что базис \textemdash\  упорядоченная система век­ торов. Это означает,
 что из одного и того же множества векторов можно составить разные базисы, по-разному нумеруя векторы.
Коэффициенты линейной комбинации, о которой идет речь
в определении базиса, называются компонентами или координатами вектора в данном базисе.\\
Векторы базиса $e_1, \ldots, e_n$ мы будем записывать в виде строки, 
а компоненты $\xi_1, \ldots, \xi_n$ вектора в базисе $\boldsymbol{e}$ -- в столбец $\begin{bmatrix} \xi_1 \\  \vdots \\ \xi_n \end{bmatrix}$ который назовем координатным столбцом вектора.\\
Разложение вектора по  базису:
 \begin{equation*}
 x = \sum \xi^ie_i =  \begin{bmatrix} p_1 &  e_2 & p_3 \end{bmatrix}  
 \begin{bmatrix} \xi_1 \\  \xi_2 \\ \xi_3 \end{bmatrix} = \boldsymbol{ e\xi}
 \end{equation*}
Из предложения \ref{sug6.5} непосредственно следует, что компоненты век­ тора в данном базисе определены однозначно.

\begin{sug}\label{sug6.6}
Координатный столбец суммы векторов ра­ вен сумме их координатных столбцов. Координатный столбец произ­ ведения вектора на число равен произведению координатного столбца данного вектора на это число.
\end{sug}
\begin{proof}
Для доказательства просто перемножим строку базиса с координатными столбцами и воспользуемся свойствами матриц.
\end{proof}

\begin{sug}\label{sug6.7}
Векторы линейно зависимы тогда и только тогда, когда линейно зависимы их координатные столбцы.
\end{sug}

\begin{sug}\label{sug6.8}
Если в линейном пространстве существует базис из n векторов, то любая система из m > n векторов линейно
зависима.
\end{sug}
\begin{proof}
Предположим, что в пространстве сущест­вует базис $e_1, \ldots, e_n$ , и рассмотрим систему векторов 
$ f_i , \ldots, f_m$, при­ чем m > n. Каждый из векторов $f_i , \ldots, f_m$ мы разложим по базису и составим матрицу из их координатных столбцов. Это матрица раз­меров m х n, и ранг ее не превосходит n. Поэтому столбцы матрицы линейно зависимы, а значит, линейно зависимы и векторы $ f_i , \ldots, f_m$
\end{proof}
\begin{theorem}\label{t6.1}
Если в линейном пространстве есть базис из n век­ торов, то и любой другой базис состоит из n векторов.
\end{theorem}

\begin{glob_def}
Линейное пространство, в котором существует базис из n векторов, называется n-мерным, а число n - размерностью пространства. Размерность пространства $\mathscr{L}$ обозначается $dim\mathscr{L}$
\end{glob_def}

В нулевом пространстве нет базиса, так как система из одно­ го нулевого вектора линейно зависима. 
Размерность нулевого прост­ранства по определению считаем равной нулю.
Может случиться, что каково бы ни было натуральное число m, в пространстве найдется m линейно
 независимых векторов. Такое пространство называется бесконечномерным. Базиса в нем не сущест­вует: 
 если бы был базис из n векторов, то любая система из n + 1 векторов была бы линейно зависимой по 
 предложению \ref{sug6.8}.
 \begin{exmpl}
 Линейное пространство функций от одной перемен­ной t, определенных и непрерывных на отрезке [0, 1] 
 является бес­конечномерным. Чтобы это проверить, достаточно доказать, что при любом m в нем существует 
 линейно независимая система из m век­торов. Зададимся произвольным числом m. Векторы нашего 
 прост­ранства - функции $t_0 = 1, t, t^2 ,\ldots, t^{m-1}$ -- линейно независимы. Действительно, 
 равенство нулю линейной комбинации этих векторов означает, что многочлен
$\alpha_0+ \alpha_1t+ \alpha_2t^2 +\ldots+\alpha_{m-1}t^{m-1}$
В n-мерном пространстве каждая упорядо­ченная линейно независимая система из n векторов есть базис.
\end{exmpl}

\begin{sug}\label{sug6.9}
Если в линейном пространстве существует базис из n векторов, то любая система из m > n векторов линейно
зависима.
\end{sug}
\begin{proof}
Очевидно.
\end{proof}

\begin{sug}\label{sug6.10}
В n-мерном пространстве каждую упорядоченную линейно независимую систему из k < n векторов можно 
до­полнить до базиса.
\end{sug}
\begin{proof}
Очевидно.
\end{proof}


\newpage
\section{4 Пункт}

\subsection{Координатное представление векторов линейного пространства и операций с ними.}

\subsection{Теорема об изоморфизме. Матрица перехода от одного базиса к другому.}

\subsection{Теорема об изоморфизме. Матрица перехода от одного базиса к другому.}

\subsection{Изменение координат при изменении базиса в линейном пространстве.}




\newpage
\section{Подпространства и способы их задания в линейном пространстве. Сумма и пересечение подпространств. Формула размерности суммы подпространств. Прямая сумма.}
\section{Линейные отображения линейных пространств и линейные преобразования линейного пространства. Ядро и образ линейного отображения. Операции над линейными преобразованиями. Обратное преобразование. Линейное пространство линейных отображений (преобразований).}
\section{Матрицы линейного отображения и линейного преобразования для конечномерных пространств. Операции над линейными преобразованиями в матричной форме. Изменение матрицы линейного отображения (пре- образования) при замене базисов. Изоморфизм пространства линейных отображений и пространства матриц.}
\section{Инвариантные подпространства линейных преобразований. Собственные векторы и собственные значения. Собственные подпространства. Линейная независимость собственных векторов, принадлежащих различным собственным значениям.}


\chapter{33rr}
paisugvpabrva pads gae adg. b fdb
\section{Ранг матрицы. Теоремы о базисном миноре. Теорема о ранге матрицы.}
[aufpiabf
\subsection{lahgo iourbg}
bpaib psd ps sipdf bd
\end{document}

%% 	matrix example
\begin{equation*}
  \begin{matrix}
	1 & 2 \\
	 3&4 
   \end{matrix} \qquad
	 \begin{bmatrix}
		p_{11} & p_{12} &  \ldots & p_{1n} \\
		p_{21} & p_{22} &  \ldots & p_{2n} \\
		\vdots & \vdots & \ddots & \vdots \\
		p_{m1} & p_{m2} & \ldots & p_{mn} 
	\end{bmatrix} 
\end{equation*}
%%